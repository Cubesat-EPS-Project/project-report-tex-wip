\thispagestyle{plain}

\begin{center}
	\Large {\bf \uppercase{ABSTRACT}}
\end{center}

\vspace{3\baselineskip}

\justifying
%We would like to express our deep and sincere gratitude to our research supervisor, \emph{\Supervisor}, for giving us the opportunity to conduct research and providing invaluable guidance throughout this work. His dynamism, vision, sincerity and motivation have deeply inspired us. He has taught us the methodology to carry out the work and to present the works as clearly as possible. It was a great privilege and honor to work and study under his guidance. 
%
%We are greatly indebted to our honorable teachers of the Department of \Department at the \University who taught us during the course of our study. Without any doubt, their teaching and guidance have completely transformed us to the persons that we are today.
%
%We are extremely thankful to our parents for their unconditional love, endless prayers and caring, and immense sacrifices for educating and preparing us for our future. We would like to say thanks to our friends and relatives for their kind support and care.
%
%Finally, we would like to thank all the people who have supported us to complete the project work directly or indirectly.

CubeSat are miniature version of satellites that offer hands-on experience to engineering students in designing, developing, testing and operating a real spacecraft system. A 1U CubeSat is a cube shaped satellite with dimensions of 10 cm x 10 cm x 10 cm and maximum mass of 1.33 kilograms. CubeSats are traditionally built from COTS-components (Commercial Off-the-Shelf) with low resources. Typically, CubeSat have limited mission time and short development and testing time. 
\\
One of the most critical aspects of the CubeSat is the Electrical Power System (EPS) since the electrical power is necessary for a CubeSat to operate. The EPS of the CubeSat consists mainly of solar cells, batteries, voltage converters and protection circuits. The EPS is responsible of providing stable power to the CubeSat subsystems.
\\
The purpose of this project is to design and implement an EPS for a CubeSat. The EPS must be able to power all subsystem components including telemetry, onboard computer, attitude determination and control system , thermal system as well as the payload while also protecting the subsystems from the over-current and over-voltage issues associated with the device failure. The system will be designed to provide power for the satellite throughout the entire orbit, even during periods of eclipse when the satellite is not able to generate power. The EPS should also provide data about voltage and current measurements, battery status, etc. to OBC (On-Board Computer).


