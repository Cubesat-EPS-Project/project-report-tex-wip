\thispagestyle{plain}
\chapter{Literature Review}
\justifying

%\begin{table}[h]
%	\begin{Center}
%	\begin{adjustbox}{width=1.02\columnwidth,center}
%		\begin{tabular}{|l|l|l|l|}
%			\hline
%			{\bf Sl.No.} &
%			{\bf Title} &
%			{\bf Author} &
%			{\bf Features} \\ \hline
%			1 &
%			\begin{tabular}[c]{@{}l@{}}A Comprehensive Review on CubeSat\\ Electrical Power System Architectures,\\ in IEEE Transactions on Power Electronics,\\ vol. 37, no. 3, pp. 3161-3177, March 2022\end{tabular} &
%			\begin{tabular}[c]{@{}l@{}}Amarendra Edpuganti ,\\ Vinod Khadkikar,\\ Mohamed Shawky El Moursi,\\ Hatem Zeineldin , Naji Al-Sayari,\\ Khalifa Al Hosani\end{tabular} &
%			\begin{tabular}[c]{@{}l@{}}Architecture with PPT and\\ regulated DC-bus was\\ selected.\end{tabular} \\ \hline
%			2 &
%			\begin{tabular}[c]{@{}l@{}}Output power analysis of Tel-USat\\ electrical power system, AIP Conference\\ Proceedings 2226, 030007 (2020)\end{tabular} &
%			\begin{tabular}[c]{@{}l@{}}Aulia Indana, Dharu Arseno,\\ Edwar, Adilla Safira\end{tabular} &
%			\begin{tabular}[c]{@{}l@{}}Centralised architecture\\ was selected.\end{tabular} \\ \hline
%			3 &
%			\begin{tabular}[c]{@{}l@{}}Comparison of Peak Power Tracking Based\\ Electric Power System Architecture for\\ CubeSats, IEEE Transactions on Industry\\ Applications, vol. 57, no. 3, pp. 2758-2768,\\ May-June 2021\end{tabular} &
%			\begin{tabular}[c]{@{}l@{}}A. Edpuganti, V. Khadkikar,\\ H. Zeineldin, M. S. E. Moursi,\\ M. Al Hosani\end{tabular} &
%			\begin{tabular}[c]{@{}l@{}}Peak power transfer\\ is preferred to\\ direct power transfer.\end{tabular} \\ \hline
%			4 &
%			\begin{tabular}[c]{@{}l@{}}A Review of Battery Technology\\ in CubeSats and Small Satellite\\ Solutions,Energies, vol. 13, 2020\end{tabular} &
%			\begin{tabular}[c]{@{}l@{}}Knap, Vaclav \& Vestergaard,\\ Lars \& Stroe, Daniel-Ioan\end{tabular} &
%			\begin{tabular}[c]{@{}l@{}}Solar cells with Li-ion\\ batteries for storage is\\ preferred.\end{tabular} \\ \hline
%			5 &
%			\begin{tabular}[c]{@{}l@{}}Review on the charging techniques\\ of a Li-Ion battery, Third International\\ Conference on Technological Advances\\ in Electrical, Electronics and Computer\\ Engineering (TAEECE), 2015\end{tabular} &
%			E. Ayoub and N. Karami &
%			Charging at 5-45\degree C \\ \hline
%		\end{tabular}
%	\end{adjustbox}
%\end{Center}
%\end{table}


 [1] shows the comparison between different CubeSat EPS architectures, from which we have identified the architecture with maximum power point tracking and a regulated DC buses to be most suitable for our application.
\\

The difference between centralized and distributed architecture was discussed in [2] and we have selected centralized architecture.
\\

As discussed in [3], solar panels operate at their most efficient points with a power point tracking algorithm, allowing the extraction of maximum power from the solar panels. Hence, peak power transfer is preferred to direct power transfer.
\\

Different battery technologies used in CubeSats were discussed in [4] and Li-ion cells were selected as the energy storage device due to their high energy density and higher number of charge discharge cycles compared to LiPo and NiMH batteries.
\\

From [5], the optimum ambient temperature for charging a Lithium ion battery is +5°C to +45°C and thus, charging is limited to this range of temperature.