\justifying
\chapter{The Electrical Power System}
The Electrical Power System (EPS) is an electronic circuit board that is designed
to supply, manage and store energy in an efficient way. The EPS must be
able to harvest energy from the solar panels and store it in the battery, as well as
delivering power to the satellite, using switch controlled converters to supply a
regulated voltage. Redundant circuitry must be present to ensure continuous and reliable operation of the satellite in case of the failure of EPS components.
\\
The output of the solar panels is first run through the power path control. While in
sunlight operation, the power path will select the voltage from the panels based on
its higher voltage. The output of the Power Path control is sent to DC-DC converters to provide 5V and 3.3V regulated DC supply for the CubeSat modules. During the eclipse, the power path will select the battery to power the circuit components.
\\
The software is implemented in order to manage the overall energy of the satellite,
regulate the converters to extract maximum power from solar panels, perform
power diagnostics, engage redundant circuitry and to communicate with the On
Board Computer. The software also employs four operating modes: Initialization
mode, Safe mode, Normal mode and Low Power Mode.
\\
\section{Requirements of EPS}

\subsection{Introduction} 

The first step before beginning the design of the EPS is to know exactly what our goals and our constraints are. This is the subject of this section.
\\ \\
There are constraints due to space environment. The EPS must withstand vacuum and wide ranges of temperatures and radiations. There are constraints linked to the launch, like accelerations, vibrations, and rules for the CubeSats in the launch vehicle. Finally, there are constraints due to the construction of CubeSat itself. The dimensions of the PCB and available volumes should be specified. 
\\ \\
The desired functionalities also guide the design. The EPS must produce enough energy to supply the CubeSat. Enough energy has to be stored to supply the satellite during the period of eclipse. The EPS must provide several power outputs with stabilized voltages. The constraints will also guide the validation tests applied to the prototypes. Since our objective is to make demo boards, all these constraints may not be fully follower while designing. 

\subsection{Constraints} 

\subsubsection{Vacuum}

The satellite is expected to be released in an elliptical orbit with an altitude between 350 and 1,000 km. At such altitudes, the atmosphere pressure can be neglected and considered as vacuum. Therefore, all components used in the satellite must withstand vacuum. The most sensitive component of the EPS is the batteries. The two threats of vacuum are:
\begin{itemize}
\item Deformations due to mechanical constraint of vacuum
\item Out gassing.
\end{itemize}

\subsubsection{Radiations}

Charged particles of solar wind, electrons, and protons, are captured by the earth magnetic field. They form the radiations belts, also known as the Van-Allen belts. There are two belts:
\begin{itemize}
\item The inner belt, between 1,000 and 15,000 km, containing high concentrations of energetic protons with energies exceeding 100 MeV and electrons in the range of hundreds of keV.
\item The outer belt, extending till 50,000 km, and consisting mainly of high energy electrons (from 0.1 to 10 MeV).
\end{itemize}

Spacecrafts need to be protected against radiations, especially if they go though the radiation belts. Trapped particles in the radiation belts and cosmic rays can cause “Single Event Phenomena” (SEP) within semiconductor devices. There are three different types of SEP:
\\ \\
\begin{itemize}
\item The Single-Event Upset (SEU)
\begin{itemize}
\item This is when a high energy particle hits a logic device and changes digitally stored data or causes a gate to open or close at the wrong time.
\end{itemize}
\item The Single-Event Latch up (SEL)
\begin{itemize}
\item The SEL is when a high energy particle directly damages the device. It can, however, be corrected if the SEL is detected and the power to the device quickly turned off, then turned back on.
\end{itemize}
\item The Single-Event Burnout (SEB)
\begin{itemize}
\item This is the case where the device is destroyed.
\end{itemize}

\end{itemize}


The radiation dose is estimated to be more than 105 Sv. A protection against SEL can be provided to the subsystems with current-limiter circuits. There is no particular protection against SEU and SEB except reducing the effect of the radiations inside the satellite with a layer of shielding aluminium (less than 2.104 Sv with 2mm of aluminium). There are components designed and/or tested to be more resistant to radiations. Such components should be used in the more critical systems of the EPS.
\\ \\
\subsubsection{Temperature}

The temperatures in space, when the satellite is turned on, will essentially depend of the thermal design. The temperature ranges are not the same everywhere in the satellite. Thermal simulations give an idea of the temperature at different points of the CubeSat. The external temperature will vary the most (from $-33^{o}C$ to $+40^{o}C)$, the temperature of the EPS will stay between $-22^{o}C$ and $+37^{o}C$, and the temperature of batteries between $-22^{o}C$ and $+37^{o}C$. The EPS must thus be able to work within these ranges.

\begin{itemize}
\item Solar cells must be selected so that they are able to work in the predicted range of $-33^{o}C$ to $+40^{o}C$.
\item The electronics on the EPS must be designed to be able to operate from $-22^{o}C$ to $+37^{o}C$ (PCB Temp). Considering wider ranges, components with an operating temperature range of at least  $40^{o}C$ to $+85^{o}C$ can be selected as ”absolute maximum rating”.
\item Lithium-Polymer batteries can withstand 0 to $+45^{o}C$ during charge and $-20^{o}C$ to $+60^{o}C$ during discharge (but with a significant loss of capacity under $0^{o}C)$. A solution must be found to maintain the batteries in these ranges of temperature.
\end{itemize}

\subsubsection{Vibrations and Accelerations during launch} 

The CubeSat will be subjected to vibrations during the launch.The effects of acceleration during launch are not to be underestimated. The CubeSat has to be able to withstand an acceleration of 15 g, even if it will probably be lower in reality. Special attention must be paid to the fixation of heavy components. Also, components could be damaged by the bending of the PCB under vibrations and accelerations. Automotive components will be chosen whenever possible.

\subsubsection{Temperatures during launch} 

The launch vehicle will pass through several atmospheric layers with specific temperatures. Inside the launch vechile, the CubeSat will endure temperatures of $-40^{o}C to 80^{o}C$ during launch. Components of the EPS have to be able to withstand such temperatures during storage (the CubeSat is inactive during launch). Components with a working temperature range of $-40^{o}C to 85^{o}C$ should not have any problems.
The batteries are still the most sensible component. Their storage temperature should stay between $-20^{o}C and 60^{o}C$. A passive solution must be found to protect them during launch. Thermal insulation and thermal inertia will certainly help.




\section[Components of EPS]{Components of EPS}
The EPS of a CubeSat can be designed with many different architectures, but some
components are common to all designs, such as:
\begin{itemize}
	\item Solar panels to harvest the energy from the Sun
	\item Battery charger to manage the charging profile of the battery
	\item Voltage regulators to feed the regulated power bus of the satellite
	\item Remove Before Flight (RBF) switches and deployment switches, to cut the
	power while the satellite is not deployed
\end{itemize}
Other components of the EPS are:
\begin{itemize}
	\item Battery and associated charging circuit
	\item Solar panels on 6 faces of the satellite
	\item MPPT converters which help optimise power collection from the sun
	\item Buck and boost converters which help provide required voltage busses for
	components of different voltages
	\item  An MCU which controls the tasks that the EPS performs and
	monitors the status of the components
	\item Over Current Protection Circuit which helps protects important components from
	high current flow
	\item  Current and Voltage sensors to keep track of their consumption.
	\item  Temperature sensors to measure battery temperature, based on which battery
	heater is used
	\item Battery heater circuit
\end{itemize}
\section[Tasks of EPS]{Tasks of EPS}
Tasks of the EPS are:
\begin{itemize}
	\item Collect housekeeping for various components associated with it, like the various
	current \& voltage sensors and the battery’s state of charge.
	\item Handle housekeeping requests and other commands from the OBC (ON/OFF requests of any subsystem by OBC).
	\item  Implement MPPT to optimize power generation.
	\item  Control the Simple Beacon (which contains only the call sign of the satellite)
	before the TTC gets switched on.
	\item  Implement a watchdog timer to keep a check on the operation of the OBC.
	\item   Take action on the basis of OCPC triggers.
	\item  Deployment of antenna at the time of satellite initialisation.
	\item  Turn on the battery heater when temperature goes below critical level
\end{itemize}