\chapter{Component Selection and Design}
\justifying
\section[Solar Panels]{Solar Panels}
 TJ Solar Cell 3G30C - Advanced is selected. This cell is a GaInP/GaAs/Ge on Ge substrate triple junction solar cell. The end-of-life version of the 3G30C solar cell offers best EOL-performance
 values. Connected to the EPS via an external bypass diode protection.
 
 Specifications:
 \begin{itemize}
 	\item Average Open Circuit Voltage: 2.7V
 	\item Maximum Power Point Voltage: 2.41V
 	\item Average Short Circuit Current: 520.2 mA
 	\item Maximum Power Point Current: 504.4mA
 \end{itemize}
It has an average efficiency of 29.8\% at 1353 $W/m^{2}$. This solar cell is excellent for space applications. 
\\
Solar panels are connected in such a way that each side has two cells connected in series.The maximum voltage developed per side is 4.4V and the maximum current that can be generated per side at peak power point is 0.5A.
Panels on opposite sides are connected in parallel.


\section[MPPT Circuit]{Maximum Power Point Tracking Circuit}
The MPPT converter connected to the solar panels increases the efficiency as the
maximum power is transferred from the radiated energy that is on the solar panels.
As each solar panel has different temperatures and incident radiance angles, the
Maximum Power Point (MPP) is also different. So each solar panel has a MPPT
converter to assure that the maximum power available at the solar panels is
transferred independently from their working power points.Since the peak power
point cannot be accurately predicted, many different algorithms exist for finding
the best approximation. The MPPT can be implemented in the EPS using one of three algorithms:
{\bf Perturb and Observe, Incremental Conductance, Constant Voltage}

The SPV1040 was chosen as the MPPT IC. It is a boost converter with duty ratio controlled by Perturb and Observe MPPT algorithm. The perturb and observe algorithm is based on monitoring either the voltage or the current supplied by the DC power source unit so that the PWM signal duty cycle is increased or decreased step-by-step according to the input power trend. This chip has inbuilt over-current protection and a cutoff mechanism if the solar panel connection is reverse-inserted to prevent damage to the IC and the external circuit.

Specifications of  SPV1040:
	\begin{itemize}
	\item Input Voltage: 0.3 - 5.5V
	\item Output Voltage: 5V
	\item Switching Frequency: 100kHz
%	\item Inbuilt over-current, temperature protection
	\item Efficiency: 95\%
\end{itemize}

\section[Battery]{Battery}
The most popular types of batteries use the following materials: Nickel Cadmium
(NiCd), Nickel Metal Hydride (NiMH), Nickel Hydrogen (NiH2), Lithium Ion
(Li-Ion) and Lithium Polymer (Li-Po). The Li-Po and Li-Ion became the standard use in space technology due to their
high energy density (Upto 200 Wh / kg on Li-Po and upto 250 Wh / kg on Li-Ion) and also due to
the number of charging cycles being as high as the NiMH, whilst presenting higher
operating temperatures. 
The Panasonic NCR 18650 GA Li-Ion cell was selected based on the calculation of EOL power, EOL efficiency and due to it's high energy density.
Specifications of Panasonic NCR 18650 GA:
\begin{itemize}
	\item Voltage: 3.7V - 4.2V
	\item Capacity: 3500mAh
	\item 1800 cycles till capacity reduces to 60\%
\end{itemize}
\section[Battery Charger]{Battery Charger}
 The battery also needs a charger to regulate its current and voltage while charging.
 BQ25302, a synchronous Buck Battery Charger IC was selected and connected in external power path mode.\\
 
 Specifications and Operating Conditions of BQ25302:
\begin{itemize}
 	\item Input Voltage: Upto 5V
 	\item Output Voltage: Upto 4.2V
 	\item Switching Frequency: 1.2MHz
 	\item Output Current: Limited to 1.2A
 	\item Efficiency: 94.3\% at 1A from 5V input
 	\item Thermistor: Semitec 103AT-2 (10\si{\kilo\ohm})
 	\item Charging Temperature: Limited between 0 - 45 $^{o}C$
 \end{itemize}



\section[BUBO]{Buck and Boost Converters}
The power conditioning is associated with regulating the voltage to accommodate
for the charging voltage and the voltages of the satellite's subsystems. In most
subsystems, the need for a specific voltage requires a regulation of either a step-up
or a step-down of the supplied voltage. It can be done by buck
convertor(step-down) and boost converter(step-up).
TPS62203 was selected as the buck converter to provide step down voltage of the DC bus to supply the 3.3V loads.\\

 Specifications and Operating Conditions of TPS62203:
\begin{itemize}
	\item Input Voltage: 3.6 - 5V
	\item Output Voltage:  3.3V
	\item Switching Frequency: 1MHz
	\item Output Current: 300mA (max.)
\end{itemize}


 LTC3426 was selected as the boost converter to provide step up voltage of the DC bus to supply the 5V loads.\\
 
  Specifications and Operating Conditions of LTC3426:
 \begin{itemize}
 	\item Input Voltage: 3.6 - 5V
 	\item Output Voltage: 5V
 	\item Switching Frequency: 1.2MHz
 	\item Output Current: 500mA (max.)
 \end{itemize}
All convertors operate in continuous conduction mode.

\section[BUBO]{Protection Circuits}
The circuit used for the protection purpose is the current limiting circuit. Unlike a
fuse that breaks a circuit connection, a current limiter only limits the current at a
predetermined level. The current limiting circuit can be as simple as a single
resistor (a passive current limiter), with the voltage drop across the resistor being
dependant on the consumed current by the load. Higher the current drawn by the
load, higher the voltage drop on that resistor. In many cases, this is not preferable.
An active current limiting circuits does not drop the voltage if the current drawn by
the load is below the allowable range. With this mechanism, all power is delivered
to the load in the normal condition. If the load tries to draw a current that is more
than allowed then the current limiting circuit will act as a resistor, controlling its
resistant value to limit the current to a predetermined level.