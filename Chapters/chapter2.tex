

\chapter{Applications}

\begin{itemize}
	\item Non-commercial broadcasting and
	and commercial broadcasting
	\item Simple to build and use
	\item Most of the components required for this circuit can be procured from your junk box.
	\item The circuit can be powered from anything between 6 to 12V DC.
	\item Useful during natural disasters to broadcast warnings or other messages when other communication techniques fail
	\item Range of antenna and frequency can be extended by choosing the wanted components
	\item Can also be used as an FM bugger circuit by taking advantage of Capture effect in FM waves.				
\end{itemize}
%\section{Set Your Report Particulars}
%
%In order to setup report particulars, Afsin would configure the report as follows:
%
%\begin{verbatim}
%% Use one: Thesis/Capstone project report/Project report
%%\def\RoportType{Capstone project report\xspace} 
%\def\RoportType{Thesis\xspace}
%%\def\RoportType{Project report\xspace}
%
%\def\ReportTitle{Blockchain Based Food Distribution in the Planet
%Mars\xspace}
%\def\Supervisor{Dr. Mohammad Shahriar Rahman\xspace} 
%\def\SupervisorPosition{Associate Professor\xspace}
%%\def\reportSubmissionDate{\today}
%\def\reportSubmissionDate{February 02, 2029}
%\def\reportSubmissionTerm{Fall 2028}
%\end{verbatim}
%\section{Set Group Members Particulars}
%According to the description given in Sec \ref{sec:how}, Afsin would configure the group members particulars as follows:
%\begin{verbatim}
%\def\numberOfAuthors{1} % write 1, 2 or 3 (depends on your group)
%%
%\def\firstAuthor{Afsin Fairuz\xspace} 
%\def\firstAuthorID{243014007\xspace}
%\def\firstAuthorFatherName{Manzurul Haque\xspace}
%\def\firstAuthorMotherName{Mansura Akhter\xspace}
%\end{verbatim}
%
%\section{Changing Chapter Title}
%In order to create a new chapter or rewrite the chapter title, you need to use \verb|\chapter{}| command. For example, this chapter starts with \verb|\chapter{How to use| \verb|this template?}| which means, the title of this chapter is ``How to use this template?''. If you want to change the title to ``Literature Review'', use the command as follows: \verb|\chapter{Literature Review}|.
%
%\section{Adding a Section}
%You can add a section using \verb|\section{}| command.
%
%\subsection{Adding a Subsection}
%You can add a subsection like this one using \verb|\subsection{}| command.
%
%\subsubsection{Adding a Sub-subsection}
%You can add a subsubsection like this one using \verb|\subsubsection{}| command.
%\section{Adding a Figure}
%Figures are often require to illustrate the concepts in scientific writing. In your report, if you want to add a figure then first import the figure in the ``Images'' folder of this project. Then add the figure to the desired location of your report using the following commands (assuming that the image you would like to add is Mars.jpg).
%\begin{verbatim}
%\begin{figure}[ht]
%    \centering
%    \includegraphics[width=0.40\textwidth]{Images/Mars.jpg}
%    \caption{Picture of the Planet Mars in natural color.}
%    \label{fig:mars}
%\end{figure}
%\end{verbatim}
%
%In the above code, a figure is added having a caption ``Pictured of the Planet Mars in natural color.''. Additionally, you should add a \verb|\label{fig:mars}| to the figure by which you can refer the figure (e.g. \verb|\ref{fig:mars}|) from the body of the text. The command \verb|\centering| is used to center align the figure.
%
%\section{Adding a Table}
%A table can be added in your documents by using tabular within a table environment. 
%
%The following is an example of a table environment:
%\begin{verbatim}
%\begin{table}[ht]
%  \centering
%  \caption{A test table.}
%  \begin{tabular}{l c c c}
%    \hline
%    Name    & Weight (lb) & Height (in) & Gender \\ \hline \hline
%    Alice   & 133         & 65          & F     \\ \hline
%    Bob     & 160         & 72          & M   \\ \hline
%    Charlie & 152         & 70          & M  \\ \hline
%    Diana   & 120         & 60          & F   \\ \hline  
%  \end{tabular}
%  \label{tab:1}
%\end{table}
%\end{verbatim}
%Next time you recompile your project, a table will be generated as shown in \ref{tab:1}.
%
%\begin{table}[ht]
%\centering
%\caption{A test table.}
%\begin{tabular}{l c c c}
%\hline
%Name    & Weight (lb) & Height (in) & Gender \\ \hline \hline
%Alice   & 133         & 65          & F     \\ \hline
%Bob     & 160         & 72          & M   \\ \hline
%Charlie & 152         & 70          & M  \\ \hline
%Diana   & 120         & 60          & F   \\ \hline  
%\end{tabular}
%\label{tab:1}
%\end{table}
%
%\section{Citing Articles}
%In order to cite an article, please copy the BibTeX for the corresponding article from Google Scholar or any digital library. A typical BibTeX of an article looks as follows:
%\begin{verbatim}
%@article{krizhevsky2012imagenet,
%  title={Imagenet classification with deep convolutional neural networks},
%  author={Krizhevsky, Alex and Sutskever, Ilya and Hinton, Geoffrey E},
%  journal={Advances in neural information processing systems},
%  volume={25},
%  pages={1097--1105},
%  year={2012}
%}    
%\end{verbatim}
%
%Get a required BibTeX and paste that in the \textit{references.bib} file of this project. You should take a note of the key to use in your report. In the above example \verb|krizhevsky2012imagenet| is the key.
%
%Next, go to the desired location in your report to insert the reference. To cite this article, please write a command as follows: \verb|\cite{krizhevsky2012imagenet}|. Next time you recompile your project, you should get a reference as follows: \cite{krizhevsky2012imagenet}
%
%Now, please go to the Bibliography section (you may jump to there by clicking on the number in green color as well) of your report where you will find bibliographic detail of your referred article.

