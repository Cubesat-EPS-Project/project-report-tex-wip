\chapter{Introduction}

\justifying
A CubeSat is a class of miniaturized satellite based around a form factor consisting
of 10 cm cubes. CubeSats have a mass of no more than 2 kg per unit, and often use
commercial off-the-shelf (COTS) components for their electronics and structure.
CubeSats are put into orbit by deployers on the International Space Station, or
launched as secondary payloads on a launch vehicle. As of August 2021, more than
1,600 CubeSats have been launched.
\\

For more than a decade, CubeSats, or small satellites, have paved the way to
low-Earth orbit for commercial companies, educational institutions, and non-profit
organizations. These small satellites offer opportunities to conduct scientific
investigations and technology demonstrations in space in such a way that is
cost-effective, timely and relatively easy to accomplish.
It give students an
experience in developing flight hardware and conducting space missions.\\

CubeSat missions benefit Earth in varying ways. From Earth imaging satellites that
help meteorologists to predict storm strengths and direction, to satellites that focus
on technology demonstrations to help define what materials and processes yield the
most useful resources and function best in a microgravity environment, the variety
of science enabled by CubeSats results in diverse benefits and opportunities for
discovery.