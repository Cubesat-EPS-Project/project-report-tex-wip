\chapter{Introduction}

\justifying
\section{Background}
Artificial satellites are foundational components of modern society. Satellites have played a huge role in communication, global positioning and the study of the solar system and beyond. The first artificial satellite, Sputnik, was launched in 1957. The launch of the first
meteorological satellite, TIROS-1, in 1960 was followed by the U.S. The first Earth-observing satellite for land-based applications was Landsat-1, which was launched in 1972. Since then many more Earth-observing satellites have been put into orbit. The whole process of designing, building, and launching a satellite-flown remote sensing system is a very lengthy and costly process. The main driving force behind the enormously costly development
of space technology has been the military, where cost is not usually a problem. With the relaxation
on restrictions for ownership and operation of Earth-observing satellites .......................




\\A CubeSat is a class of miniaturized satellite based around a form factor consisting
of 10 cm cubes. CubeSats have a mass of no more than 2 kg per unit, and often use
commercial off-the-shelf (COTS) components for their electronics and structure.
CubeSats are put into orbit by deployers on the International Space Station, or
launched as secondary payloads on a launch vehicle. As of August 2021, more than
1,600 CubeSats have been launched.
\\

For more than a decade, CubeSats, or small satellites, have paved the way to
low-Earth orbit for commercial companies, educational institutions, and non-profit
organizations. These small satellites offer opportunities to conduct scientific
investigations and technology demonstrations in space in such a way that is
cost-effective, timely and relatively easy to accomplish.
It give students an
experience in developing flight hardware and conducting space missions.\\

CubeSat missions benefit Earth in varying ways. From Earth imaging satellites that
help meteorologists to predict storm strengths and direction, to satellites that focus
on technology demonstrations to help define what materials and processes yield the
most useful resources and function best in a microgravity environment, the variety
of science enabled by CubeSats results in diverse benefits and opportunities for
discovery.

\section{Objective}

%To design and implement a fully autonomous power generation, storage and distribution
%system for a CubeSat
The aim of this project is to determine requirements for a typical CubeSat Electrical Power System (EPS) and develop a working prototype of the EPS for a CubeSat.

\section{Literature Review}
\justifying
[1] shows the comparison between different CubeSat EPS architectures, from which we have identified the architecture with maximum power point tracking and a regulated DC buses to be most suitable for our application.
\\

The difference between centralized and distributed architecture was discussed in [2] and we have selected centralized architecture.
\\

As discussed in [3], solar panels operate at their most efficient points with a power point tracking algorithm, allowing the extraction of maximum power from the solar panels. Hence, peak power transfer is preferred to direct power transfer.
\\

Different battery technologies used in CubeSats were discussed in [4] and Li-ion cells were selected as the energy storage device due to their high energy density and higher number of charge discharge cycles compared to LiPo and NiMH batteries.
\\

From [5], the optimum ambient temperature for charging a Lithium ion battery is +5°C to +45°C and thus, charging is limited to this range of temperature.