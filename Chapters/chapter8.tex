\chapter{Methodology}
\justifying
\section{Identifying Power Requirements}
Before designing the EPS, the power requirements of the various subsystems of the CubeSat has to be identified. A power budget has to be prepared accounting all the energy, voltage and current requirements of the subsystems. The orbital parameters at which the CubeSat might be operating should also be considered. The orbital altitude, period and eclipse time and the daylight time has to be identified and documented. After this, the peak power budget has to be calculated and total energy and power demands are to be found out.
\section{Literature Review}
In order to select the suitable architecture and topologies, literature study has to be conducted. Various articles regarding the implementation of CubeSats and EPS were studied and the findings were recorded.
\section{Architecture Design and Topology selection}
The design of EPS starts with the selection of appropriate EPS architecture based on the comparison of overall efficiency, battery size, and reliability. The EPS design is critical for CubeSat mission success, therefore selection of proper EPS architecture is one of the important steps. Different standard EPS architectures are classified on the basis of various topologies like dc-bus voltage regulation, interface of PV panels, location of power converters, and number of conversion stages. The necessary topology has to be selected based on the demands and constraints.
\section{Forming Specifications}
After deciding upon a suitable architecture, the specifications of various components of the EPS has to be finalised. This includes deciding the number of required power converters and their input and output parameters, deciding the number, size and type of battery for energy storage and the characteristics of the solar panels and specifications of the MPPT device.
\section{Design and simulation}
Suitable ICs able to perform the various functions of different components in an EPS have to be identified. The ICs must be suitable for operation in outer space. After selecting the ICs, the design of them are to be completed and necessary schematics and PCB design has to be completed. Also, the circuits obtained have to be verified with the help of simulation results.
\section{Procurement of components}
The components which were finalised has to be procured. Surface Mount components are preferred due to the space constraints, also the selected components must be applicable in outer space applications.
\section{Fabrication and Testing}
The components have to be soldered into the PCB and the results are to be observed. Initially, each component maybe developed individually and tested before optimizing the entire circuit into a single, centralized form. 